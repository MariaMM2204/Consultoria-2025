% Options for packages loaded elsewhere
\PassOptionsToPackage{unicode}{hyperref}
\PassOptionsToPackage{hyphens}{url}
%
\documentclass[
  11,
]{article}
\usepackage{amsmath,amssymb}
\usepackage{iftex}
\ifPDFTeX
  \usepackage[T1]{fontenc}
  \usepackage[utf8]{inputenc}
  \usepackage{textcomp} % provide euro and other symbols
\else % if luatex or xetex
  \usepackage{unicode-math} % this also loads fontspec
  \defaultfontfeatures{Scale=MatchLowercase}
  \defaultfontfeatures[\rmfamily]{Ligatures=TeX,Scale=1}
\fi
\usepackage{lmodern}
\ifPDFTeX\else
  % xetex/luatex font selection
\fi
% Use upquote if available, for straight quotes in verbatim environments
\IfFileExists{upquote.sty}{\usepackage{upquote}}{}
\IfFileExists{microtype.sty}{% use microtype if available
  \usepackage[]{microtype}
  \UseMicrotypeSet[protrusion]{basicmath} % disable protrusion for tt fonts
}{}
\makeatletter
\@ifundefined{KOMAClassName}{% if non-KOMA class
  \IfFileExists{parskip.sty}{%
    \usepackage{parskip}
  }{% else
    \setlength{\parindent}{0pt}
    \setlength{\parskip}{6pt plus 2pt minus 1pt}}
}{% if KOMA class
  \KOMAoptions{parskip=half}}
\makeatother
\usepackage{xcolor}
\usepackage[margin=1in]{geometry}
\usepackage{longtable,booktabs,array}
\usepackage{calc} % for calculating minipage widths
% Correct order of tables after \paragraph or \subparagraph
\usepackage{etoolbox}
\makeatletter
\patchcmd\longtable{\par}{\if@noskipsec\mbox{}\fi\par}{}{}
\makeatother
% Allow footnotes in longtable head/foot
\IfFileExists{footnotehyper.sty}{\usepackage{footnotehyper}}{\usepackage{footnote}}
\makesavenoteenv{longtable}
\usepackage{graphicx}
\makeatletter
\newsavebox\pandoc@box
\newcommand*\pandocbounded[1]{% scales image to fit in text height/width
  \sbox\pandoc@box{#1}%
  \Gscale@div\@tempa{\textheight}{\dimexpr\ht\pandoc@box+\dp\pandoc@box\relax}%
  \Gscale@div\@tempb{\linewidth}{\wd\pandoc@box}%
  \ifdim\@tempb\p@<\@tempa\p@\let\@tempa\@tempb\fi% select the smaller of both
  \ifdim\@tempa\p@<\p@\scalebox{\@tempa}{\usebox\pandoc@box}%
  \else\usebox{\pandoc@box}%
  \fi%
}
% Set default figure placement to htbp
\def\fps@figure{htbp}
\makeatother
\setlength{\emergencystretch}{3em} % prevent overfull lines
\providecommand{\tightlist}{%
  \setlength{\itemsep}{0pt}\setlength{\parskip}{0pt}}
\setcounter{secnumdepth}{5}
\usepackage{setspace}
% this is a lorem ipsum generator for adding dummy texts
\usepackage{lipsum}
\usepackage{tocloft}
% to make the first rows bold in tables
\usepackage{longtable}
\usepackage{tabu}
\usepackage{booktabs}
% this makes list of figures appear in table of contents
\usepackage[nottoc]{tocbibind}

% for passing temporary notes
\usepackage{todonotes}

\usepackage{morefloats}
\usepackage{float}

% highlighting
\usepackage{soul}

% referencing mutliple things with a single command - \cref
\usepackage{hyperref}
\usepackage{cleveref}


% this makes dots in table of contents
\renewcommand{\cftsecleader}{\cftdotfill{\cftdotsep}}
% to change the title of contents
% \renewcommand{\contentsname}{Whatever}

% line numbers for review purposes
% this package might not be available in default latex installation 
% get it by 'sudo tlmgr install lineno'
%\usepackage{lineno}
%\linenumbers

% this allows checkmarks in the file
\usepackage{amssymb}
\DeclareUnicodeCharacter{2714}{\checkmark}

% to be able to include latex comments
\newenvironment{dummy}{}{}
\usepackage{bookmark}
\IfFileExists{xurl.sty}{\usepackage{xurl}}{} % add URL line breaks if available
\urlstyle{same}
\hypersetup{
  hidelinks,
  pdfcreator={LaTeX via pandoc}}

\author{}
\date{\vspace{-2.5em}}

\begin{document}

\onehalfspacing
\pagenumbering{gobble} 
%\begin{titlepage} 
\begin{center}
 \vspace*{2\baselineskip} 
\includegraphics[width=0.25\textwidth]{UAB.png}\\
\vspace*{5\baselineskip}
\normalsize{Universitat Autònoma de Barcelona}\\
\normalsize{\textbf{Consultoria Estadística}}
 \LARGE{\textbf{}}\\
 \vspace*{5\baselineskip} 
\Large{\textbf{LOTERIA DE NADAL}}\\ 
\vspace*{2\baselineskip} 
\Large{Maria Marín Méndez (1668394) \\ Andrea Acuña Villagaray (1639232)}\\
\vspace*{3\baselineskip}
\Large{\textbf{29 Gener, 2026}}\\ 
\end{center} 
% \end{titlepage} 
\doublespacing 
\hypersetup{linkcolor = black} 
\newpage 
\pagenumbering{roman} 
\tableofcontents 
\addcontentsline{toc}{section}{\contentsname} 
\newpage % list of figures have to be added manually to table of contents 
\doublespacing 
\newpage 
\pagenumbering{arabic} 
\hypersetup{linkcolor = blue}

\section{Resum}\label{resum}

El Sorteig Extraordinari de Nadal, també conegut com a Loteria de Nadal,
és un dels sorteigs més populars que se celebra a Espanya cada desembre
des de l'any 1812.

L'objectiu principal d'aquest estudi es centra en l'anàlisi exhaustiva
dels resultats històrics del sorteig, així com el desenvolupament d'un
model capaç de predir els números guanyadors de la Loteria de Nadal.

\newpage

\section{Descripció del conjunt de
dades}\label{descripciuxf3-del-conjunt-de-dades}

\subsection{Introducció de les dades}\label{introducciuxf3-de-les-dades}

En aquest repositori es recull el desenvolupament del Treball Final de
Consultoria Estadística 2025, centrat en l'anàlisi exhaustiva de la
Loteria de Nadal des de la seva vessant més tècnica. L'objectiu no és
sols descriure el sorteig, sinó avaluar amb rigor estadístic si la
variabilitat dels resultats històrics (des del 2000 fins a l'anys 2025)
respon purament a l'atzar o si existeixen anomalies mesurables en
l'homogeneïtat del sistes,

A través de metodologies de web scraping, tests d'independència \ldots..

La Loteria de Nadal no és sols un sorteig de boles; és l'unic moment de
l'any en què un país sencer es posa d'acord per ignorar les lleis de
l'estadística. Des d'un punt de vista matemàtic, és un ``impost a
l'esperança'', però des del punt de vista de lesa dades, és un
ecosistema fascinant.

\textbf{L'arquitectura del ``GORDO''}

La loteria no ven números a l'atzar, sinó que segueix una jerarquia
rígida que determina quants diners es mouen i quina probabilitat real
tens de guanyar.

El sistema treballa amb un ventall de 100.000 números (des del 00000
fins al 99999). Això ens deixa una probabilitat de guanyar el primer
premi amb un sol dècim de un 0.001\%.

Per al 2025, SELAE (Societat Estatal Loteries i Apostes de l'Estat) va
emetre 197 sèries de cada número. Això significa que de cada número n'hi
ha 197 billets idèntic repartits per tot l'estat. On cada sèroe es
dovodeox en 10 dècims, el que ens dona un total de 1970 dècims per cada
número.

\begin{itemize}
\tightlist
\item
  Preu del dècim: 20€
\item
  Recaptació potencila, si es ven tot, el sorteig mou 3940 milions
  d'euros. D'aquests, el 705 es destina a premis.
\end{itemize}

El que veiem a la tele és folkore, però el que hi ha dins dels bombos és
física pura. com ens explica el professor Badiella, el sistema està
dissenyat per se homogeni.

Hi ha 100.000 boles al bombo gran, totes fetes de gusta deboic, amb un
diàmetre de 3cm i un pes unificat. On per evitar que el pes de la
pintura alteri el camí de la bola cap a la trompeta, els números estan
impresos amb l'àser. Així s'elimina la teoria que els números amb més
``tinta'', com el 88.888, pesen més que el 11.111.

Per a realtizar el sorteig es fan servis dos bombos que giren
simultàniament: un per als números i un latre amb les 1807 boles de
premis. fins que el bombo petit no queda buit, el sorteig no es dona per
acabat.

\textbf{El repartiment del ``Pastís''}

Tot i que el focus està en el Gordo, la realitat és que el sorteig és
una pluja fina de premis petits que serveixen per ``alimentar
l'esperança'' de cara a l'anys següent.

\begin{longtable}[]{@{}llll@{}}
\toprule\noalign{}
Premi & Import per dècim & Boles premiades & Probabilitat \\
\midrule\noalign{}
\endhead
\bottomrule\noalign{}
\endlastfoot
1r premi (``el Gordo'') & 400.000€ & 1 & 0,001\% \\
2n premi & 125.000€ & 1 & 0,001\% \\
3r premi & 50.000€ & 1 & 0,001\% \\
4rt premi & 20.000€ & 2 & 0,002\% \\
5é premi & 6.000€ & 8 & 0,008\% \\
La Pedrea & 100€ & 1.794 & 1,794\% \\
Reintegrament & 20€ & 1 de cada 10 xifres & 10,00\% \\
\end{longtable}

Aleshores amb aquesta informació la probabilitat de que et toqui el
Gordo és la mateixa que la de qualsevol número, 0,001\%. Gairebé el 99\%
dels números que reben premi directe del bombo són pedres de 100€ (1.794
de 1.807). Per altra banda, tenim un 10\% de possibilitats de recuperar
20€ si l'última xifra del teu número coincideix amb la del Gordo.

Hem de recordar que per als tres primers premis, el premi realment és
menor, ja que Hisenda entra al joc aplicant un 20\% d'impost a qualsevol
premi que superi els 40.000€. Això vol dir que, el guanyador del primer
premi realment es de 328.000€, ja que els premiers 40.000€ estan exempts
i es paga 20\% dels 360.000€ restant. El segon premi sseria de 108.000€
i del trecer 48.000€.

Per on podem començar l'anàlisi?

\begin{enumerate}
\def\labelenumi{\arabic{enumi}.}
\tightlist
\item
  Històric de terminacions: quins números han sortir més.
\item
  Geografia de la sort: Analitzar si és veritat que el premi toca més a
  Madrid o Barcelona, o si és simplement perquè s'hi venene més dècims.
\item
  Simulació de Montecarlo (proposta gemini): crear un script en R que
  simuli 10.000 sortejos per veure queantes vegades guanyaries el Gordo
  si juguessis el mateix número durant 100 anys.
\end{enumerate}

\section{Exploració de les dades}\label{exploraciuxf3-de-les-dades}

A continuació presenterem diverses estadístiques descriptives així com
visualitzacions gràfiques que permetràn poder observar el comportament
de les dades.

\subsection{Lectura de dades}\label{lectura-de-dades}

En primer lloc, realitzarem una lectura inicial de tots els números
premiats en el sorteig de la Loteria de Nadal corresponents al període
entre 2000 i 2025. Les dades utilitzades provenen dels resultats
oficials publicats un cop finalitzat cada sorteig.

A l'endemà del sorteig, es publica el següent arxiu per comprovar si el
número ha sigut premiat:

\begin{figure}[h]
  \centering
  \includegraphics[width=0.7\textwidth]{loteria.png}
  \caption{Resultats Sorteig loteria de nadal 2023}
\end{figure}

El codi utilitzat per a la lectura i el processament de les dades
s'implementarà a través d'un script extern, el qual importarem
mitjançant la funció d'R source().

A la taula següent mostrarem alguns exemples dels premis per veure amb
quines dades treballarem:

La taula mostra un extracte dels números premiats corresponents al
sorteig del 2024. Cada fila representa un número premia, identificat per
la seva xifra, juntament amb informació addicional com la lletra
associada.

Hi han tres tipus de lletra que ens podem trobar dins del nostre
dataset:

\begin{enumerate}
\def\labelenumi{\arabic{enumi}.}
\item
  Lletra \textbf{t}:
\item
  Lletra \textbf{a}:
\item
  Lletra \textbf{b}:
\item
  Lletra \textbf{c}:
\end{enumerate}

Finalment, també obtindrem els diners premiats, la categoria del premi,
l'any i en quina moneda s'ha repertit els premis. Ja que ens els primers
anys tindrem la moneda com a pessetes espanyoles.

\subsection{Anàlisi descriptiva}\label{anuxe0lisi-descriptiva}

(Posar taula: Estadística descriptiva conjunt de dades)

Aquest conjunt de dades compta amb un dataset corresponent a cada any
analitzat, amb una mida mitjana de 4000 números premiats per any. Així
que tindrem una mostra suficient per dur a terme el nostra anàlisis.

\subsection{Visualització Gràfica}\label{visualitzaciuxf3-gruxe0fica}

A continuació visualitzarem diverses representacions gràfiques.

(Posar: Gràfic distribució números)

(Posar: Última xifra números premiats)

(Fer: Quantitat premis per categoria (decena, centena\ldots), modo serie
temporal amb punts)

(Probar Heatmap a veure si es veu bé)

(Fer: Interpretació)

\subsection{Discussió possibles valors
atípics}\label{discussiuxf3-possibles-valors-atuxedpics}

Explicar com tractarem lo de que van cambiant el sorteig + PESETAS /
EUROS

Que fem???

\section{Aplicació d'algorismes per la
modelització}\label{aplicaciuxf3-dalgorismes-per-la-modelitzaciuxf3}

\subsection{Metodologia}\label{metodologia}

Hi ha diversos algorismes que podem aplicar per desenvolupar un model
predictiu, en aquest treball utilitzarem l'algorisme anomenat x.

X és una tècnica on \ldots. D'aquesta manera el model podrà\ldots{}

x calcula \ldots. i \ldots.

\subsection{Entranament del model amb el conjunt de
dades}\label{entranament-del-model-amb-el-conjunt-de-dades}

Per entrenar el model amb \ldots{} hem \ldots.. D'aquesta manera, podrem
analitzar\ldots.

Hem dividir el dataset en conjunts d'entrenament i validació, amb les
dades de l'any 2000 al 2024 dels resultats per entrenar el model i
l'últim concurs realitzat al 2025 per a fer la validació. Així podrem
observar el rendiment del model.

Un cop preparada la mostra d'entrenament,\ldots. Ja realitzat tot el
prepossessing, ja podem definir i entrenar el model.

L'estructura d'aquest es veu tal que:

\section{Resultats}\label{resultats}

\subsection{Presentació dels
resultats}\label{presentaciuxf3-dels-resultats}

\subsection{Mètriques d'avaluació}\label{muxe8triques-davaluaciuxf3}

\newpage

\section{Conclusions}\label{conclusions}

\newpage

\section{Annexos}\label{annexos}

A continuació, introduirem el directori de Git-Hub on podràs trobar tot
el codi d'R utilitzat:

\url{https://github.com/MariaMM2204/Consultoria-2025}

\newpage

\section{References}\label{references}

\end{document}
